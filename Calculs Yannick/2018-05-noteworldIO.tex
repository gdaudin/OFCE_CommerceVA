\documentclass[A4]{article}

\usepackage{amsmath}
\renewcommand{\vec}{\boldsymbol}
\usepackage{setspace}
\doublespacing

\title{Proxy for the import contents}
\author{Yannick Kalantzis}


\begin{document}

\maketitle 

2 countries: Home and Foreign. Foreign variables are starred.

Vectors of production: $Q, Q^*$.
Denote $\cal Q = \left[
  \begin{array}{l}
    Q \\ Q^*
  \end{array}
\right]$ the vector of production in both countries.

Direct domestic requirement matrices are $A, A^*$. Import requirement
matrices are $B, B^*$. The world input-output table is $\cal A = \left(
  \begin{array}{ll}
    A & B^* \\ B & A^*
  \end{array}
\right)$.

Sectoral vectors of domestic consumption are $C$ for
domestically produced consumption and $C^*$ for imported consumption,
with total consumption normalized to 1. Denote $\cal C = \left[
  \begin{array}{l}
    C \\ C^*
  \end{array}
\right]$.

We have $\cal Q = \cal A \cal Q + \cal C$ so $\cal Q = \cal L\cal C$,
with ${\cal L} = (I-{\cal A})^{-1}$ the world Leontiev inverse ($I$ is
the identity matrix).

Denote the submatrices of $\cal L$ as:
$\cal L = \left(
  \begin{array}{ll}
    L & M^* \\ M & L^*
  \end{array}
\right)$.

Then we have
$ \left[
  \begin{array}{l}
    Q \\ Q^*
  \end{array}
\right] = \left(
  \begin{array}{ll}
    L & M^* \\ M & L^*
  \end{array}
\right)\left[
  \begin{array}{l}
    C \\ C^*
  \end{array}
\right]$.
Sectoral domestic production associated to a unit of domestic consumption is given by $Q =
LC+M^*C^*$. 

If I get it right, the vector of import content of domestic
consumption should be 
\begin{equation*}
  \underbrace{C^*}_\text{direct} +
  B\Bigl (\underbrace{LC}_\text{\shortstack{indirect due to\\ domestic
      production}}+\underbrace{M^*C^*}_\text{\shortstack{indirect due
      to\\ domestic production of \\ inputs for Foreign}}\Bigr).
\end{equation*}

The import content itself is the scalar product of $\vec 1$ (a vector
of ones) and the vector of import content: $\vec 1\cdot C^*+\vec
1\cdot B(LC+M^*C^*)$.

Next, denote $\tilde L = (I-A)^{-1}$ the inverse Leontiev of domestic
production alone (ignoring the world input-output linkages). This is
what we would use with traditional matrices from one country only. We
can decompose the import content in the following way:

\begin{equation}
\label{eq1}
  \underbrace{\vec 1 \cdot C^*}_\text{direct} +
  \vec 1 \cdot B\Bigl (\underbrace{\tilde L C}_\text{\shortstack{indirect due
      to\\ domestic I/O linkages}} + \underbrace{(L-\tilde L)
    C}_\text{\shortstack{global value chain \\ of domestic
      production}}+\underbrace{M^*C^*}_\text{\shortstack{indirect due
      to\\ domestic production of \\ inputs for Foreign}} 
  \Bigr).
\end{equation}
The first two terms can be computed with I/O matrices from 1 country
only. The last two terms require the full world I/O matrix and reflect
global value chains. I think this expression generalizes easily to the
case of many countries (the square matrices $B$ and $M^*$ should simply become rectangular).

Can we make a crude approximation of this formula using only simple
ratios?
Denote $\delta = \vec 1\cdot C^*$ the share of direct imports in
consumption, $\beta = \vec 1 \cdot B$ the share of imported inputs in
total domestic production. Let $\alpha$ the share of domestic inputs in total
domestic production and $\mu$ the total share of inputs in total
production: we have $\alpha+\beta=\mu$ and $\mu$ is typically $\approx
1/2$. Finally, define $\tilde \beta = \beta/(1-\mu)$ the share of
imports in total inputs used by the domestic economy.

Assuming homogeneous sectors, we get a very crude approximation of the
second term:\footnote{The matrix $B$ becomes $\beta$, $\tilde L$
  becomes $1/(1-\alpha)$, $C$ becomes $1-\delta$.}
$\beta(1-\delta)/(1-\alpha)=\beta(1-\delta)/(1-\mu+\beta)\approx\beta(1-\delta)/(1-\mu)\approx\tilde\beta(1-\delta)$
to the first order in $\beta$. My
guess is that the total import content should be highly correlated
with
\begin{equation*}
  \delta + (1-\delta)\tilde{\beta}.
\end{equation*}

If you run a regression of the import content on $\delta$ and
$(1-\delta)\tilde\beta$, you should get a decent fit. The coefficient
of $\delta$ should be slightly higher than 1, the difference with 1
potentially reflecting global value chain effect captured by the last
term of (\ref{eq1}). The coefficient of $(1-\delta)\tilde\beta$ might
also be different from 1, perhaps capturing the third term of (\ref{eq1}).
\end{document}
