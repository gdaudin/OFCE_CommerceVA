\documentclass[11pt,a4paper]{article}
\usepackage[T1]{fontenc}
%\usepackage[latin1]{inputenc}
%\usepackage{amssymb,amsmath,a4wide}
\usepackage[utf8]{inputenc}
\usepackage{amssymb,amsmath}
\usepackage{nicefrac}
\usepackage[pdftex]{graphicx}
\usepackage{ctable}
\usepackage{amsmath}
%\usepackage{threeparttable} %na
%\usepackage{tabu} %na
\usepackage{tabularx}
\usepackage{subfig}
\usepackage{rotating}
\usepackage{longtable}
%\usepackage[table]{xcolor} % clash with floatrow
\usepackage{xcolor} 
\usepackage{threeparttable}
%\usepackage[multiple]{footmisc} %na
\usepackage{bm}
\usepackage{fancybox}
%\usepackage{harvard}
\usepackage{geometry}         % Definir les marges
\geometry{verbose,a4paper,tmargin=1in,bmargin=1in,lmargin=1in,rmargin=1in}
\usepackage{setspace}
%\usepackage{ccaption}
\usepackage[colorlinks=true,citecolor=black, urlcolor=black, linkcolor=black]{hyperref}
\usepackage{url}
\newcommand{\email}[1]{\href{mailto:#1}{\nolinkurl{#1}}}
%\usepackage[longnamesfirst]{natbib}
\usepackage{natbib}
\usepackage[french, english]{babel}  % Placez ici une liste de langues, la derniere etant la langue principale
\usepackage{caption}
\usepackage{floatrow}
\usepackage{lscape}
\usepackage{afterpage}
\usepackage{supertabular}    %na            %  mettre pour les grands tableaux en formant paysage. marche avec \begin{landscape}
% style de la biblio : necessaire pour utiliser BibTex, necessite le deuxieme fichier exemple.bib


\bibliographystyle{elsarticle-harv}
\newcommand{\tqdl}{\textquotedblleft}
\newcommand{\tqdr}{\textquotedblright}
\linespread{1.2}

\begin{document}
\title{Answers to Hubert Escaith\\
\vspace{1cm}
}
\vspace{1cm}
\date{\today}
\author{Guillaume Daudin\thanks{PSL, Universit\'e Paris-Dauphine,Sciences Po, OFCE. E-mail: \email{guillaume.daudin@dauphine.psl.eu}}}
%\vspace*{\fill}
\maketitle

\section{Why not based on VA trade ?}
Why don't we simply : \\
1. Compute the origin of the VA content of each good \\
2. Study how the price evolve following a shock on the price of VA in a country or another ?

Intuition: \\
That would not do because the price of, e.g. French VA does not change for everybody.

Doubt: is that enough an argument ?
1 sector, 2 countries
\subsection{Evolution of VA price}
\begin{gather*}
A=\left(\begin{matrix}a_{1,1}&a_{1,2}\\a_{2,1}&a_{2,2}\end{matrix}\right)
\\
I-A=\left(\begin{matrix}1-a_{1,1}&-a_{1,2}\\-a_{2,1}&1-a_{2,2}\end{matrix}\right)
\\
\left(I-A\right)^{-1}=\frac{1}{\left(1-a_{1,1}\right)\left(1-a_{2,2}\right)-a_{2,1}a_{1,2}}\left(\begin{matrix}1-a_{2,2}&a_{1,2}\\a_{2,1}&1-a_{1,1}\end{matrix}\right) =z.\left(\begin{matrix}1-a_{2,2}&a_{1,2}\\a_{2,1}&1-a_{1,1}\end{matrix}\right) \\ 
=\left(\begin{matrix}u&v\\w&x\end{matrix}\right)
\\
\text{French demand shares}=d=\left(\begin{matrix}1-f\\f\end{matrix}\right) \\
\left(I-A\right)^{-1}d=\left(\begin{matrix}u-uf+vf\\w-wf+xf\end{matrix}\right)
\end{gather*}

Donc, en cas de choc $c$ pour le prix de la va dans le pays étranger (en monnaie française), on peut écrire un vecteur de choc : $C=\left(0,c\right)$.  les prix varient tout d'abord de  $CA$, puis $CA^2$, etc. Donc le vecteur de choc $S$ (en monnaie française) est : 


\begin{equation*}
S=C+CA+CA^2...=C(I-A)^{-1}=\left(\begin{matrix}cw  &   cx\end{matrix}\right)
\end{equation*}

To measure the effect on French consumption prices, we do a weighted sum of these effects.

\begin{equation}
\bar{s}=c.\left[\left(1-f\right)w+xf\right]=c.\frac{\left(1-f\right)a_{2,1}+f\left(1-a_{1,1}\right)}{\left(1-a_{1,1}\right)\left(1-a_{2,2}\right)-a_{2,1}a_{1,2}}
\end{equation}

If each nation's production only uses national inputs, we have a plausible :
\begin{equation*}
\bar{s}=c.\frac{f}{1-a_{2,2}}
\end{equation*}

\subsection{Exchange rate shock}

Using the notations in the paper...

\begin{gather*}
C=\left(0,\frac{-c_\$}{1+c_\$}\right)=\left(0,-c\right)
\\
C_\$=\left(c_\$,0\right)
\\
\tilde{C}_\$=\left(0,-c_\$\right)
\\
\hat{C}_\$=\left(\frac{c_\$}{1+c_\$},0\right)=\left(c,0\right)
\\
\cal B = \left(\begin{matrix}0&a_{1,2}\\0&0\end{matrix}\right)
\\
\tilde{\cal B} = \left(\begin{matrix}0&0\\a_{2,1}&0\end{matrix}\right)
\end{gather*}

Hence

\begin{gather*}
S =\left(0,c\right)+\left[\left(0,-c.a_{1,2}\right)+\left(c.a_{2,1},0\right)\right]*\left(\begin{matrix}u&v\\w&x\end{matrix}\right)
\\
=\left(0,c\right)+\left(c.a_{2,1},-c.a_{1,2}\right)*\left(\begin{matrix}u&v\\w&x\end{matrix}\right)
\\
=\left(0,c\right)+\left(u.c.a_{2,1}-w.c.a_{1,2},v.c.a_{2,1}-x.c.a_{1,2}\right)
\\
=\left(u.c.a_{2,1}-w.c.a_{1,2},c+v.c.a_{2,1}-x.c.a_{1,2}\right)
\end{gather*}
and
\begin{gather*}
\bar{s}=\left(u.c.a_{2,1}-w.c.a_{1,2},c+v.c.a_{2,1}-x.c.a_{1,2}\right).\left(\begin{matrix}1-f\\f\end{matrix}\right)
\\
\bar{s}=c\left[f\left(1+v.a_{2,1}-x.a_{1,2}\right)+\left(1-f\right)\left(u.a_{2,1}-w.a_{1,2}\right)\right]
\end{gather*}


If each nation's production only uses national inputs, we have a plausible

\begin{gather*}
\bar{s}=c.f
\end{gather*}

This seems to confirm that the exchange rate shock is not the same as the VA price shock.

\subsection{Residual issue}

Starting from 4.2 in the paper
\begin{gather*}
E1=C
\\
E2=C.\tilde{\cal B}=\left(0,c\right).\left(\begin{matrix}0&0\\a_{2,1}&0\end{matrix}\right)=\left(c.a_{2,1},0\right)
\end{gather*}

\begin{gather*}
E1.HC = \left(0,c\right).\left(\begin{matrix}1-f\\f\end{matrix}\right)=f.c
\\
E2.HC=\left(c.a_{2,1},0\right).\left(\begin{matrix}1-f\\f\end{matrix}\right)=c.a_{2,1}.\left(1-f\right)
\\ 
\bar{s}-E1.HC-E2.HC=
\\
c\left[f\left(1+v.a_{2,1}-x.a_{1,2}\right)+\left(1-f\right)\left(u.a_{2,1}-w.a_{1,2}\right)\right]-c\left(f+a_{2,1}.\left(1-f\right)\right)
\\
=c\left[a_{2,1}\left(\left(f-1\right)\left(1-u\right)+vf\right)+a_{1,2}\left(\left(1-f\right)w-x\right)\right]
\end{gather*}

Easy : we can normalize the shock $c$ to 1. 

\begin{gather}
residual=a_{2,1}\left(\left(f-1\right)\left(1-u\right)+vf\right)+a_{1,2}\left(\left(1-f\right)w-x\right)
\end{gather}

How can we continue to show that this thing does not depend on the openness/size of the economy?
\\
 Idea : Hypothesis that (but it does not help) $$a_{2,1}=a_{1,2}$$ and (does not help)$$a_{1,1}=a_{2,2}$$ ? 
More interesting $$\frac{a_{1,1}}{a_{2,1}}=\frac{a_{2,2}}{a_{1,2}}=\frac{1-f}{f}$$ 
and 
$$a_{1,1}+a_{2,1}=a$$

So...
$$a_{1,1}=(1-f)a$$
$$a_{2,1}=fa$$
Then 

\begin{gather}
residual=fa\left(\left(1-f\right)\left(1+u\right)+vf\right)+a_{1,2}\left(\left(1-f\right)w-x\right) \\
FAUX=z.\left[fa\left(\left(1-f\right)\left(1+(1-a_{2,2})\right)+a_{1,2}f\right)+a_{1,2}\left(\left(1-f\right)af-(1-f)a\right)\right] \\
= z\left[fa\left(\left(1-f\right)\left(2-a_{2,2})\right)+a_{1,2}f\right)-a_{1,2}a\left(1-f\right)^2\right] \\
= \frac{fa\left(\left(1-f\right)\left(2-a_{2,2})\right)+a_{1,2}f\right)-a_{1,2}a\left(1-f\right)^2}{\left(1-a_{1,1}\right)\left(1-a_{2,2}\right)-a_{2,1}a_{1,2}} \\
= \frac{fa\left(\left(1-f\right)\left(2-a_{2,2})\right)+a_{1,2}f\right)-a_{1,2}a\left(1-f\right)^2}{\left(1-a(1-f)\right)\left(1-a_{2,2}\right)-afa_{1,2}}
\end{gather}

Pour dériver, je développe: 
\begin{gather*}
residual=\frac{f^2a\left(-2+a_{2,2}\right)+fa\left(2-a_{2,2}+2a_{1,2}\right)-aa_{1,2}}{fa\left(1-a_{2,2}-a_{1,2}\right)+1-a+a_{2,2}(1+a)}
\end{gather*}

Le signe de cette dérivée est celui de : 
\begin{gather*}
\left(
	2fa\left(
		-2+a_{2,2}\right)
	+a\left(
		2-a_{2,2}+2a_{1,2}\right)
	\right)
\left(
	fa\left(
		1-a_{2,2}-a_{1,2}\right)
+1-a+a_{2,2}(1+a)\right) \\
-a\left(
	1-a_{2,2}-a_{1,2}\right)
\left(
	f^2a\left(
		-2+a_{2,2}\right)
	+fa\left(
		2-a_{2,2}+2a_{1,2}\right)
	-aa_{1,2}\right)
\end{gather*}

Et alors là... Aucune idée du signe de cette expression...
\begin{gather*}
\left(
	fa\left(
		1-a_{2,2}-a_{1,2}\right)
+1-a+a_{2,2}(1+a)\right) > 0
\\
a\left(
	1-a_{2,2}-a_{1,2}\right) >0
\\
\left(
	2fa\left(
		-2+a_{2,2}\right)
	+a\left(
		2-a_{2,2}+2a_{1,2}\right)
	\right) = a\left[\left(1-a_{2,2}\right)\left(1-2f\right)\right]+2a{2,1}	>0
\end{gather*}


\end{document}