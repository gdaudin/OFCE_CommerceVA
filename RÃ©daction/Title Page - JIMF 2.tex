\documentclass[11pt,a4paper]{paper}
\usepackage[T1]{fontenc}
\usepackage[utf8]{inputenc}
\usepackage{amssymb,amsmath}
\usepackage{nicefrac}
\usepackage[pdftex]{graphicx}
\usepackage{ctable}
\usepackage{amsmath}
\usepackage{tabularx}
\usepackage{subfig}
\usepackage{rotating}
\usepackage{longtable}
\usepackage{xcolor} 
\usepackage{floatrow}
\usepackage{threeparttable}
\usepackage{bm}
\usepackage{fancybox}
\usepackage{geometry}
\geometry{verbose,a4paper,tmargin=3.5cm,bmargin=3.5cm,lmargin=2.5cm,rmargin=2.5cm}
\usepackage{setspace}
\usepackage[colorlinks=true,citecolor=black, urlcolor=black, linkcolor=black]{hyperref}
\usepackage{url}
\newcommand{\email}[1]{\href{mailto:#1}{\nolinkurl{#1}}}
\usepackage[french, english]{babel}
\usepackage{lscape}
\usepackage{afterpage}
\usepackage{supertabular}
\usepackage{caption}
\usepackage[round,sort]{natbib}
\newcommand{\tqdl}{\textquotedblleft}
\newcommand{\tqdr}{\textquotedblright}
\DeclareUnicodeCharacter{FB01}{fi}
\DeclareUnicodeCharacter{FB00}{ff}
\usepackage{setspace}
\setstretch{1.5}
\pagenumbering{arabic} 



% !TeX spellcheck = en_US


\begin{document}
\title{Estimating the elasticity of consumer prices to the exchange rate: an accounting approach	\thanks{We thank Marion Cochard, Pavel Diev, Hubert Escaith, Guillaume Gaulier, Yannick Kalantzis, Guy Levy-Rueff, Sébastien Miroudot, Jean-François Ouvrard, François de Soyres and two anonymous referees from the ECB and from the JIMF for comments and suggestions, as well as participants at Banque de France seminars. Any remaining errors are ours. Online Appendix and all programs are available at \url{https://github.com/gdaudin/OFCE_CommerceVA}. Data and results are available upon request. Declarations of interest: none}\\
\vspace{1cm}
}
\vspace{1cm}
\date{\today}
\author{
	Hadrien Camatte\thanks{Banque de France. E-mail: \email{hadrien.camatte@banque-france.fr}}
	\and
	Guillaume Daudin\thanks{Université Paris-Dauphine, PSL University, CNRS, 8007, IRD, 260, LEDa, DIAL, 75016, Paris, France. Sciences Po, OFCE, 75007, Paris. Corresponding author. E-mail: \email{guillaume.daudin@dauphine.psl.eu}}
	\and
	Violaine Faubert\thanks{Banque de France. E-mail: \email{violaine.faubert@banque-france.fr}}
	\and
	Antoine Lalliard\thanks{Banque de France. E-mail: \email{antoine.lalliard@banque-france.fr}}
	\and
	Christine Rifflart\thanks{Sciences Po, OFCE. E-mail: \email{christine.rifflart@sciencespo.fr}}
}
\maketitle


\begin{abstract}
{\small \noindent
We analyse the elasticity of the household consumption expenditure (HCE) deflator to the exchange rate, using world input-output tables (WIOTs) from 1995 to 2019. 
In line with the existing literature, we find a modest output-weighted elasticity of around 0.1.
This elasticity is stable over time but heterogeneous across countries, ranging from 0.05 to 0.22. 
Such heterogeneity mainly reflects differences in foreign product content of consumption and intermediate products. 
Direct effects through imported consumption and intermediate products entering domestic production explain most of the transmission of an exchange rate appreciation to domestic prices.
By contrast, indirect effects linked to participation in global value chains play a limited role. 
Our results are robust to using four different WIOTs datasets. 
As WIOTs are data-demanding and available with a lag of several years, we extrapolate a reliable year- and country-specific estimate of the HCE deflator elasticity from 2015 onwards, using trade data and GDP statistics. 
}
\end{abstract}

\end{document}
