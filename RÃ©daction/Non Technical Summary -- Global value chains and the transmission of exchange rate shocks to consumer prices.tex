\documentclass[12pt,a4paper]{article}
\usepackage[T1]{fontenc}
%\usepackage[latin1]{inputenc}
%\usepackage{amssymb,amsmath,a4wide}
\usepackage[utf8]{inputenc}
\usepackage{amssymb,amsmath}
\usepackage{nicefrac}
\usepackage[pdftex]{graphicx}
\usepackage{ctable}
\usepackage{amsmath}
%\usepackage{threeparttable} %na
%\usepackage{tabu} %na
\usepackage{tabularx}
\usepackage{subfig}
\usepackage{rotating}
\usepackage{longtable}
%\usepackage[table]{xcolor} % clash with floatrow
\usepackage{xcolor} 
\usepackage{floatrow}
\usepackage{threeparttable}
%\usepackage[multiple]{footmisc} %na
\usepackage{bm}
\usepackage{fancybox}
%\usepackage{harvard}
\usepackage{geometry}         % Definir les marges
\geometry{verbose,a4paper,tmargin=1in,bmargin=1in,lmargin=1in,rmargin=1in}
\usepackage{setspace}
%\usepackage{ccaption}
\usepackage[colorlinks=true,citecolor=black, urlcolor=black, linkcolor=black]{hyperref}
\usepackage{url}
\newcommand{\email}[1]{\href{mailto:#1}{\nolinkurl{#1}}}
\usepackage[french, english]{babel}  % Placez ici une liste de langues, la derniere etant la langue principale
\usepackage{lscape}
\usepackage{afterpage}
\usepackage{supertabular}    %na            %  mettre pour les grands tableaux en formant paysage. marche avec \begin{landscape}
% style de la biblio : necessaire pour utiliser BibTex, necessite le deuxieme fichier exemple.bib
\usepackage{caption}
%\usepackage[longnamesfirst]{natbib}
\usepackage{natbib}
%\bibliographystyle{elsarticle-harv}
\bibliographystyle{plainnat}
\newcommand{\tqdl}{\textquotedblleft}
\newcommand{\tqdr}{\textquotedblright}
\DeclareUnicodeCharacter{FB01}{fi}
\DeclareUnicodeCharacter{FB00}{ff}
%\linespread{1.2}
\usepackage{setspace}
\doublespacing



% !TeX spellcheck = en_US


\begin{document}
\title{Global value chains and the transmission of exchange rate shocks to consumer prices \\ Non-technical summary
\vspace{1cm}
}
\vspace{1cm}
\date{\today}
\author{
	Hadrien Camatte\thanks{Banque de France. E-mail: \email{hadrien.camatte@banque-france.fr}}
	\and
	Guillaume Daudin\thanks{Université Paris-Dauphine, PSL University, CNRS, 8007, IRD, 260, LEDa, DIAL, 75016, Paris, France. Sciences Po, OFCE, 75007, Paris. Corresponding author. E-mail: \email{guillaume.daudin@dauphine.psl.eu}}
	\and
	Violaine Faubert\thanks{Banque de France, ECB. E-mail: \email{violaine.faubert@ecb.europa.eu}}
	\and
	Antoine Lalliard\thanks{Banque de France. E-mail: \email{antoine.lalliard@banque-france.fr}}
	\and
	Christine Rifflart\thanks{Sciences Po, OFCE. E-mail: \email{christine.rifflart@sciencespo.fr}}
}
%\vspace*{\fill}
\maketitle

%Nouvelle intro avril 2021 ou non technical summary
This paper focuses on the elasticity of the household consumption expenditure (HCE hereafter) deflator to the exchange rate. 
%Data
We analyse the composition and determinants of the HCE deflator elasticity using world input-output tables (WIOT hereafter) covering twenty years of data, from 1995 to 2015.
For the sake of robustness, we use several datasets (WIOD, two distinct releases of the OECD TiVA database and the MRIO database developed by the Asian Development Bank). 
%Method
We perform an accounting exercise based on information contained in the WIOT with large matrices inversion. 
%We assume a Cobb-Douglas production framework where firms have a simple cost-minimising behaviour and work in a partial equilibrium setting.
We make two additional assumptions to simplify our computations. 
First, we assume a full exchange rate pass-through to import prices. 
Hence, wo do not consider the fact that the pass-through might be incomplete, as suggested by a large body of literature (see for example \cite{Berman2012}).
Using alternative exchange rate pass-through assumptions would thus entail lower elasticity values. 
We also assume that all pricings occur using the currency of the producing country (in line with the ‘producer currency pricing’ paradigm). 
However, a large body of empirical literature suggests that the vast majority of trade is invoiced in a small number of ‘dominant currencies,’ with the U.S. dollar playing a major role. 
The ‘dominant currency paradigm’ (\cite{Gopinath2020}) implies that for non-U.S. countries, the exchange rate pass-through into import prices (in home currency) should be high and driven by the dollar exchange rate as opposed to the bilateral exchange rate, whereas for the U.S. the pass-through into import prices should be low.
Despite these simplifying assumptions, our approach helps identifying which countries and sectors are under pressure to adjust their prices when subject to an exchange rate variation.\\
%We also assume that the exchange rate shock is proportional across sectors. \\
%Variation élasticite dans le temps et l'espace
Our contribution to the literature is fourfold. 
First, we document the evolution of the elasticity of the HCE deflator to the exchange rate.
In line with the existing literature, we find a rather modest elasticity. 
The elasticity ranges from 0.09 to 0.13 depending on the dataset used.
These values are likely an upper bound. 
Indeed, the elasticity is estimated under the simplifying assumption of full exchange rate pass-through (ERPT hereafter) to import prices and under the producer pricing assumption.
However, a large body of literature suggests that the ERPT is incomplete, even in the long run, as a result of slow nominal price adjustments or the pricing-to-market behaviour of firms (see \cite{Ozyurt2016} for a discussion of the literature).
For example, the pass-through depends on the intensity of competition in domestic markets: while an exchange rate appreciation lowers the price of imported inputs, a firm with limited competitive pressure may avail of greater profit margins rather than reduce prices in an effort to maintain its market share.
Hence, pricing-to-market strategies of exporters aiming to defend their market shares would imply a lower exchange rate pass-through.
Similarly, local currency pricing, which is a particular form of pricing-to-market, refers to the situations where exporting firms adapt their mark-ups depending on the destination market to offset exchange rate movements. 
In large and attractive markets, competitive pressures may push producers to adopt local currency pricing strategies in order to limit the negative volume effect of a currency appreciation. 
Hence, under the local currency pricing paradigm, prices are assumed to be sticky in the currency of the destination
market.
Settlement and invoicing of imports in the domestic currency is another factor weakening the elasticity of domestic prices to exchange rate movements.
Hence, using alternative pricing assumptions would entail different elasticity values. \\
We find that the elasticity of the HCE deflator to the exchange rate has remained broadly stable over the past two decades.
Using the WIOD database, which covers a sample of 43 countries, we find that the mean output-weighted elasticity of the HCE deflator increased from 0.08 in 2000 to 0.09 in 2008. 
After peaking in 2008, the elasticity slightly declined between 2009 and 2015. 
Our finding concurs with the literature.
Using comprehensive measures of global value chain integration, \cite{Timmer2016} find that the expansion of global value chain has slowed since the 2008-2009 Great Recession.
By contrast, we find that the HCE deflator elasticity is heterogeneous across countries. 
The elasticity ranges from 0.05 to 0.35, reflecting different degrees of openness to trade. 
In the euro area, the elasticity of the HCE deflator ranges from 0.07 in Italy to 0.18 in Ireland. % impact du choc en euro, pas de la monnaie ficitive car on va supprimer cette reference
The elasticity is close to 0.10 in Italy, France, Germany, Spain, Portugal and Greece. 
It is twice higher for small open economies like Luxembourg, Malta, Slovakia and Ireland.\\
% Impact par secteur
Second, building on sectoral data, we examine which sectors experience higher spillovers from an exchange rate appreciation. 
To do so, we focus on the main components of the HCE deflator, i.e. manufacturing goods, services, food and energy. 
We analyse the contribution of these different products to the HCE deflator elasticity.
Non-energy industrial goods explain the bulk of the total HCE deflator elasticity to the exchange rate. 
Services also play a significant role, especially in advanced economies. 
Although services are mainly produced domestically and do not rely much on imported inputs, they make up a substantial share of total consumption. \\
%Decomposition et canaux de transmission
Third, we analyse the determinants of the HCE elasticity and the role of global value chains in the transmission of an exchange rate appreciation to the HCE deflator. 
When production processes are global, an exchange rate appreciation impacts the HCE deflator through four distinct channels: \textit{i)} the prices of imported final goods sold directly to domestic consumers;
\textit{ii)} the prices of imported inputs entering domestic production; 
%Hence, a currency appreciation reduces the price of imported inputs, and thus further reduces domestic consumer prices. 
\textit{iii)} the price of exported inputs feeding through imported foreign production;
\textit{iv)}, changes in domestic and foreign production costs in turn pass through to the price of inputs for domestic and foreign goods and cause further production costs variations through input-output linkages.\\
We find that the first two channels explain three quarters of the transmission of an exchange rate appreciation to domestic prices.
By contrast, the last two channels, which reflect the impact of global value chains, play a more limited role, with marked across-countries heterogeneity.
Hence, only one-fourth of the HCE elasticity to the exchange rate is attributable to participation in global value chains.\\
%Extrapolation
Fourth, we show that a precise assessment of the HCE deflator elasticity to the exchange rate can be estimated without resorting to WIOT. \\
The construction of World Input-Output tables is data-demanding and WIOTs are typically released with a lag of several years.
As a result, WIOT are not available for the most recent years. The latest dataset dates to 2015. 
To address this gap, we estimate the HCE elasticity from 2016 onwards using GDP statistics and trade data on consumption and intermediates.\\

\end{document}
