\documentclass[12pt,a4paper]{article}
\usepackage[T1]{fontenc}
%\usepackage[latin1]{inputenc}
%\usepackage{amssymb,amsmath,a4wide}
\usepackage[utf8]{inputenc}
\usepackage{amssymb,amsmath}
\usepackage{nicefrac}
\usepackage[pdftex]{graphicx}
\usepackage{ctable}
\usepackage{amsmath}
%\usepackage{threeparttable} %na
%\usepackage{tabu} %na
\usepackage{tabularx}
\usepackage{subfig}
\usepackage{rotating}
\usepackage{longtable}
%\usepackage[table]{xcolor} % clash with floatrow
\usepackage{xcolor} 
\usepackage{floatrow}
\usepackage{threeparttable}
%\usepackage[multiple]{footmisc} %na
\usepackage{bm}
\usepackage{fancybox}
%\usepackage{harvard}
\usepackage{geometry}         % Definir les marges
\geometry{verbose,a4paper,tmargin=1in,bmargin=1in,lmargin=1in,rmargin=1in}
\usepackage{setspace}
%\usepackage{ccaption}
\usepackage[colorlinks=true,citecolor=black, urlcolor=black, linkcolor=black]{hyperref}
\usepackage{url}
\newcommand{\email}[1]{\href{mailto:#1}{\nolinkurl{#1}}}
\usepackage[french, english]{babel}  % Placez ici une liste de langues, la derniere etant la langue principale
\usepackage{lscape}
\usepackage{afterpage}
\usepackage{supertabular}    %na            %  mettre pour les grands tableaux en formant paysage. marche avec \begin{landscape}
% style de la biblio : necessaire pour utiliser BibTex, necessite le deuxieme fichier exemple.bib
\usepackage{caption}
%\usepackage[longnamesfirst]{natbib}
\usepackage{natbib}
%\bibliographystyle{elsarticle-harv}
\bibliographystyle{plainnat}
\newcommand{\tqdl}{\textquotedblleft}
\newcommand{\tqdr}{\textquotedblright}
\DeclareUnicodeCharacter{FB01}{fi}
\DeclareUnicodeCharacter{FB00}{ff}
%\linespread{1.2}
\usepackage{setspace}
\doublespacing



% !TeX spellcheck = en_US


\begin{document}
\title{Global value chains and the transmission of exchange rate shocks to consumer prices \\ Non-technical summary
\vspace{1cm}
}
\vspace{1cm}
\date{\today}
\author{
	Hadrien Camatte\thanks{Banque de France. E-mail: \email{hadrien.camatte@banque-france.fr}}
	\and
	Guillaume Daudin\thanks{Université Paris-Dauphine, PSL University, CNRS, 8007, IRD, 260, LEDa, DIAL, 75016, Paris, France. Sciences Po, OFCE, 75007, Paris. Corresponding author. E-mail: \email{guillaume.daudin@dauphine.psl.eu}}
	\and
	Violaine Faubert\thanks{Banque de France, ECB. E-mail: \email{violaine.faubert@ecb.europa.eu}}
	\and
	Antoine Lalliard\thanks{Banque de France. E-mail: \email{antoine.lalliard@banque-france.fr}}
	\and
	Christine Rifflart\thanks{Sciences Po, OFCE. E-mail: \email{christine.rifflart@sciencespo.fr}}
}
%\vspace*{\fill}
\maketitle

Inflation rates in advanced economies have been a puzzle since the 2008 financial crisis. While the Phillips curve suggests an inverse relationship between the unemployment rate and inflation, inflation remained much higher than suggested by the high level of economic slack during the economic downturn (2009-2011). From 2012 onwards, inflation remained subdued despite the pick-up in economic activity. The apparent disconnection between inflation and its traditional domestic drivers triggered a debate on the potential growing role of global determinants. \\*
We contribute to this debate by investigating the impact of global value chains on inflation dynamics over the past twenty years. We use three sectoral world input-output datasets covering a sample of 43 to 65 economies. We pay particular attention to inflation dynamics in the euro area.\\
Participation in global value chains strengthens cross-country linkages via trade in intermediate inputs. When value chains are global, fluctuations in exchange rates affect household consumption expenditure (HCE hereafter) deflator through four distinct channels: i) changes in the prices of imported final goods sold directly to domestic consumers; ii) changes in the prices of imported inputs embedded in domestic production; iii) changes in the price of exported inputs embedded in imported foreign production and finally, iv), changes in domestic and foreign production costs passing through to the price of inputs for domestic and foreign goods and causing further production costs variations through input-output linkages.\\
Assuming a Cobb-Douglas production framework where firms follow a simple cost-minimising behaviour, we compute the partial-equilibrium effects of an exchange rate shock on consumer prices. The partial equilibrium approach is useful for identifying which countries and sectors are under pressure to adjust their prices when subject to exchange rate shocks. We use world input-output tables covering twenty years of data (from 1995 to 2015).\\
We analyse the impact of exchange rate shocks on the main components of consumer prices (manufacturing goods, services, food and energy) and on the prices of imported and domestic final goods. We confirm the importance of global value chains in explaining inflation dynamics.\\*
We find that the elasticity of the household consumption expenditure deflator to exchange rate shocks ranges from 0.05 to 0.35, depending on the openness rate of each economy. In the euro area, the elasticity of the HCE deflator to changes in the euro exchange rate ranges from 0.07 to 0.18. Such heterogeneity in the pass-through of exchange rate shocks to domestic prices adds to the challenges faced by the European Central Bank in stabilising prices throughout the monetary union. \\
We show that the direct impact (through imported final goods) and domestic input-output linkages (i.e. domestic final goods produced using foreign inputs) account for most of the propagation of an exchange rate shock to domestic prices. First-round effects explain three-quarters of the propagation of exchange rate shocks to domestic prices. By contrast, we find a limited role for the second-round effects, i.e. the additional transmission of lower domestic input prices (in the case of an appreciation of the national currency) to other sectors of the domestic economy and other countries occurring during subsequent production cycles. Domestic core inflation (i.e. inflation excluding food and energy) accounts for a significant share of the total elasticity, mainly reflecting the weight of domestic services and non-energy industrial goods in total consumption. \\
The construction of World Input-Output tables (WIOT hereafter) is data-demanding and WIOTs are typically released with a lag of several years. As a result, at the time of writing, the latest WIOT dates back to 2015. To address this gap, we take advantage of more up-to-date GDP and trade data to approximate the partial equilibrium impact of an exchange rate shock on the HCE deflator for recent years. We thus provide a tool for computing up-to-date estimates of the elasticity of consumer prices to exchange rate shocks.\\*
We analyse time variation in the elasticities over the past two decades. On a sample of 43 countries, the mean output-weighted elasticity of the HCE deflator to exchange rate shocks increased from 0.08 in 2000 to 0.09 in 2008. The elasticity declined in subsequent years. Extrapolations suggest that this decline was reversed in 2017 and 2018. 



\end{document}
