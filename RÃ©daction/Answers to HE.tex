\documentclass[11pt,a4paper]{article}
\usepackage[T1]{fontenc}
%\usepackage[latin1]{inputenc}
%\usepackage{amssymb,amsmath,a4wide}
\usepackage[utf8]{inputenc}
\usepackage{amssymb,amsmath}
\usepackage{nicefrac}
\usepackage[pdftex]{graphicx}
\usepackage{ctable}
\usepackage{amsmath}
%\usepackage{threeparttable} %na
%\usepackage{tabu} %na
\usepackage{tabularx}
\usepackage{subfig}
\usepackage{rotating}
\usepackage{longtable}
%\usepackage[table]{xcolor} % clash with floatrow
\usepackage{xcolor} 
\usepackage{floatrow}
\usepackage{threeparttable}
%\usepackage[multiple]{footmisc} %na
\usepackage{bm}
\usepackage{fancybox}
%\usepackage{harvard}
\usepackage{geometry}         % Definir les marges
\geometry{verbose,a4paper,tmargin=1in,bmargin=1in,lmargin=1in,rmargin=1in}
\usepackage{setspace}
%\usepackage{ccaption}
\usepackage[colorlinks=true,citecolor=black, urlcolor=black, linkcolor=black]{hyperref}
\usepackage{url}
\newcommand{\email}[1]{\href{mailto:#1}{\nolinkurl{#1}}}
\usepackage[french, english]{babel}  % Placez ici une liste de langues, la derniere etant la langue principale
\usepackage{lscape}
\usepackage{afterpage}
\usepackage{supertabular}    %na            %  mettre pour les grands tableaux en formant paysage. marche avec \begin{landscape}
% style de la biblio : necessaire pour utiliser BibTex, necessite le deuxieme fichier exemple.bib
\usepackage{caption}
%\usepackage[longnamesfirst]{natbib}
\usepackage{natbib}
\bibliographystyle{elsarticle-harv}
\newcommand{\tqdl}{\textquotedblleft}
\newcommand{\tqdr}{\textquotedblright}
\linespread{1.2}

\begin{document}
\title{Answers to Hubert Escaith\\
\vspace{1cm}
}
\vspace{1cm}
\date{\today}
\author{Guillaume Daudin\thanks{PSL, Universit\'e Paris-Dauphine,Sciences Po, OFCE. E-mail: \email{gdaudin@mac.com}}}
%\vspace*{\fill}
\maketitle

\section{Why not based on VA trade ?}
Why don't we simply : \\
1. Compute the origin of the VA content of each good \\
2. Study how the price evolve following a shock on the price of VA in a country or another ?

Intuition: \\
That would not do because the price of, e.g. French VA does not change for everybody.

Doubt: is that enough an argument ?
2 sectors, 2 countries
\begin{eqnarray}
A=\left(\begin{matrix}a_{1,1}&a_{2,1}\\a_{1,2}&a_{2,2}\end{matrix}\right)
\\
I-A=\left(\begin{matrix}1-a_{1,1}&-a_{2,1}\\-a_{1,2}&1-a_{2,2}\end{matrix}\right)
\\
\left(I-A\right)^{-1}=\frac{1}{\left(1-a_{1,1}\right)\left(1-a_{2,2}\right)-a_{1,2}a_{2,1}}\left(\begin{matrix}1-a_{2,2}&a_{2,1}\\a_{1,2}&1-a_{1,1}\end{matrix}\right) =\left(\begin{matrix}u&v\\w&x\end{matrix}\right)
\\
\text{French demand shares}=d=\left(\begin{matrix}1-f\\f\end{matrix}\right) \\
\left(I-A\right)^{-1}d=\left(\begin{matrix}u-uf+vf\\w-wf+xf\end{matrix}\right)
\end{eqnarray}

Donc, en cas de choc $c$ pour le prix de la va dans le pays étranger (en monnaie française), on peut écrire un vecteur de choc : $C=\left(0,c\right)$.  les prix varient tout d'abord de  $CA$, puis $CA^2$, etc. Donc le vecteur de choc $S$ (en monnaie française) est : 


\begin{eqnarray}
S=C+CA+CA^2...=C(I-A)^{-1}=\left(\begin{matrix}cw  &   cx\end{matrix}\right)
\end{eqnarray}

To measure the effect on French consumption prices, we do a weighted sum of thess effects.

\begin{eqnarray}
c\left[\left(1-f\right)w+xf\right]=c.\frac{\left(1-f\right)a_{1,2}+f\left(1-a_{1,1}\right)}{\left(1-a_{1,1}\right)\left(1-a_{2,2}\right)-a_{1,2}a_{2,1}}
\end{eqnarray}

\end{document}