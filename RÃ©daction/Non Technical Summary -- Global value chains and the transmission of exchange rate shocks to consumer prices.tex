\documentclass[11pt,a4paper]{article}
\usepackage[T1]{fontenc}
%\usepackage[latin1]{inputenc}
%\usepackage{amssymb,amsmath,a4wide}
\usepackage[utf8]{inputenc}
\usepackage{amssymb,amsmath}
\usepackage{nicefrac}
\usepackage[pdftex]{graphicx}
\usepackage{ctable}
\usepackage{amsmath}
%\usepackage{threeparttable} %na
%\usepackage{tabu} %na
\usepackage{tabularx}
\usepackage{subfig}
\usepackage{rotating}
\usepackage{longtable}
%\usepackage[table]{xcolor} % clash with floatrow
\usepackage{xcolor} 
\usepackage{floatrow}
\usepackage{threeparttable}
%\usepackage[multiple]{footmisc} %na
\usepackage{bm}
\usepackage{fancybox}
%\usepackage{harvard}
\usepackage{geometry}         % Definir les marges
\geometry{verbose,a4paper,tmargin=1in,bmargin=1in,lmargin=1in,rmargin=1in}
\usepackage{setspace}
%\usepackage{ccaption}
\usepackage[colorlinks=true,citecolor=black, urlcolor=black, linkcolor=black]{hyperref}
\usepackage{url}
\newcommand{\email}[1]{\href{mailto:#1}{\nolinkurl{#1}}}
\usepackage[french, english]{babel}  % Placez ici une liste de langues, la derniere etant la langue principale
\usepackage{lscape}
\usepackage{afterpage}
\usepackage{supertabular}    %na            %  mettre pour les grands tableaux en formant paysage. marche avec \begin{landscape}
% style de la biblio : necessaire pour utiliser BibTex, necessite le deuxieme fichier exemple.bib
\usepackage{caption}
%\usepackage[longnamesfirst]{natbib}
\usepackage{natbib}
%\bibliographystyle{elsarticle-harv}
\bibliographystyle{plainnat}
\newcommand{\tqdl}{\textquotedblleft}
\newcommand{\tqdr}{\textquotedblright}
\DeclareUnicodeCharacter{FB01}{fi}
\DeclareUnicodeCharacter{FB00}{ff}
%\linespread{1.2}
\usepackage{setspace}
\setstretch{1.5}
%\doublespacing



% !TeX spellcheck = en_US


\begin{document}
\title{Estimating the impact of exchange rate fluctuations on domestic prices: an accounting approach \\ Non-technical summary
\vspace{1cm}
}
\vspace{1cm}
\date{\today}
\author{
	Hadrien Camatte\thanks{Banque de France. E-mail: \email{hadrien.camatte@banque-france.fr}}
	\and
	Guillaume Daudin\thanks{Université Paris-Dauphine, PSL University, CNRS, 8007, IRD, 260, LEDa, DIAL, 75016, Paris, France. Sciences Po, OFCE, 75007, Paris. Corresponding author. E-mail: \email{guillaume.daudin@dauphine.psl.eu}}
	\and
	Violaine Faubert\thanks{Banque de France, ECB. E-mail: \email{violaine.faubert@ecb.europa.eu}}
	\and
	Antoine Lalliard\thanks{Banque de France. E-mail: \email{antoine.lalliard@banque-france.fr}}
	\and
	Christine Rifflart\thanks{Sciences Po, OFCE. E-mail: \email{christine.rifflart@sciencespo.fr}}
}
%\vspace*{\fill}
\maketitle

%Nouvelle intro avril 2021 ou non technical summary
This paper focuses on the impact of exchange rate fluctuations on domestic prices. 
%Data
We use several datasets covering most advanced and emerging economies, from 1995 to 2018. 
%Method
We perform an accounting exercise based on information contained in world input-output tables with large matrices inversion.
Our accounting approach helps identifying which countries and sectors are under pressure to adjust their prices when subject to an exchange rate variation.\\
%We assume a Cobb-Douglas production framework where firms have a simple cost-minimising behaviour and work in a partial equilibrium setting.
%First, we assume that exchange rate fluctuations completely pass-through to import prices. 
%Hence, wo do not consider the fact that the pass-through might be incomplete, as suggested by a large body of literature (see for example \cite{Berman2012}).
%Using alternative exchange rate pass-through assumptions would thus entail lower elasticity values. 
%Second, we assume that all pricings occur using the currency of the producing country (in line with the ‘producer currency pricing’ paradigm). 
%However, a large body of empirical literature suggests that the vast majority of trade is invoiced in a small number of ‘dominant currencies,’ with the U.S. dollar playing a major role. The ‘dominant currency paradigm’ (\cite{Gopinath2020}) implies that for non-U.S. countries, the exchange rate pass-through into import prices (in home currency) should be high and driven by the dollar exchange rate as opposed to the bilateral exchange rate, whereas for the U.S. the pass-through into import prices should be low.
%We also assume that the exchange rate shock is proportional across sectors. 
%Variation élasticite dans le temps et l'espace
Our main findings are fourfold. 
First, we document the evolution of the impact of exchange rate variations on consumer prices.
In line with the existing literature, we find that a 1\% appreciation of the domestic currency decreases domestic consumer prices by arround 0.10\%. 
This modest estimate is likely an upper bound. 
Indeed, we make a number of assumptions to simplify our computations.
First, we assume that exchange rate fluctuations completely pass-through to import prices.
However, a large body of literature suggests that the pass-through is incomplete, even in the long run, as a result of slow nominal price adjustments or the pricing-to-market behaviour of firms (see \cite{Ozyurt2016} for a discussion of the literature).
For example, the pass-through depends on the intensity of competition in domestic markets: while an exchange rate appreciation lowers the price of imported inputs, a firm with limited competitive pressure may avail of greater profit margins rather than reduce prices in an effort to maintain its market share.
Hence, pricing-to-market strategies of exporters aiming to defend their market shares would imply a lower exchange rate pass-through.
In addition, we work under the producer pricing assumption.
However, in large and attractive markets, competitive pressures may push producers to adopt local currency pricing strategies in order to limit the negative volume effect of a currency appreciation.
Local currency pricing, which is a particular form of pricing-to-market, refers to the situations where exporting firms adapt their mark-ups depending on the destination market to offset exchange rate movements. 
Hence, under the local currency pricing paradigm, prices are assumed to be sticky in the currency of the destination market.
Settlement and invoicing of imports in the domestic currency is another factor weakening the elasticity of domestic prices to exchange rate movements.
Hence, using alternative pricing assumptions would entail different elasticity values. \\
We find that that the impact of exchange rate variations on consumer prices has remained broadly stable over the past two decades. 
Using the WIOD database, which covers a sample of 43 countries, we find that the mean GDP-weighted elasticity of consumer prices to the exchange rate inreased from 0.08 in 2000 to 0.09 in 2008. 
After peaking in 2008, the elasticity slightly declined between 2009 and 2015. \\
%Using the MRIO database, which provides data up to 2018, we find that the elasticity has bounced back from 2015 onwards. However, the MRIO database suffers from data quality issues in 2018 and is likely unreliable for this year (see online Appendix F).\\
%Our finding concurs with the literature. Using comprehensive measures of global value chain integration, \cite{Timmer2016} find that the expansion of global value chain has slowed since the 2008-2009 Great Recession.
Second, we document that the impact of a 1\% exchange rate fluctuation on domestic prices is heterogeneous across countries. 
The impact ranges from 0.05\% to 0.22\%, reflecting different degrees of openness to trade. 
In the euro area, the impact ranges from 0.07\% in Italy to 0.18\% in Ireland. % impact du choc en euro, pas de la monnaie ficitive car on va supprimer cette reference
While he elasticity is close to 0.10 in Italy, France, Germany, Spain, Portugal and Greece, it is twice higher for small open economies like Luxembourg, Malta, Slovakia and Ireland.
We also estimate the impact of an appreciation of the US dollar on its trading partners.
The highest impacts are obseverd for the US neighbouring countries and/or major trading partners (Canada, Mexico and Ireland).\\
% Impact par secteur
Second, building on sectoral data, we examine which sectors experience higher spillovers from an exchange rate appreciation. 
We analyse the contribution of manufacturing goods, services, food and energy to the total impact.
Non-energy industrial goods explain the bulk of the impact of an exchange rate variation on consumer prices. 
Services also play a significant role, especially in advanced economies. 
Although services are mainly produced domestically and do not rely much on imported inputs, they make up a substantial share of total consumption. \\
%Decomposition et canaux de transmission
Third, we analyse the role of global value chains in the transmission of an exchange rate appreciation to consumer prices. 
When production processes are global, an exchange rate appreciation impacts consumer prices through four distinct channels: \textit{i)} the prices of imported final goods sold directly to domestic consumers;
\textit{ii)} the prices of imported inputs entering domestic production; 
%Hence, a currency appreciation reduces the price of imported inputs, and thus further reduces domestic consumer prices. 
\textit{iii)} the price of exported inputs feeding through imported foreign production;
\textit{iv)}, changes in domestic and foreign production costs in turn pass through to the price of inputs for domestic and foreign goods and cause further production costs variations through input-output linkages.\\
We find that the first two channels explain three quarters of the transmission of an exchange rate appreciation to domestic prices.
By contrast, the last two channels, which reflect the impact of participation in global value chains, play a more limited role, with marked across-countries heterogeneity.\\
%Extrapolation
Fourth, we show that a precise assessment of the impact of exchange rate variations on consumer prices can be estimated without resorting to world input output tables. \\
The construction of World Input-Output tables is data-demanding and WIOTs are typically released with a lag of several years.
To fill the data gap, we compute a reliable estimate of the impact of exchange rate variations on consumer prices using up-to-date GDP and trade statistics. \\

\end{document}
